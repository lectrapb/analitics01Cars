% Options for packages loaded elsewhere
\PassOptionsToPackage{unicode}{hyperref}
\PassOptionsToPackage{hyphens}{url}
%
\documentclass[
]{article}
\usepackage{amsmath,amssymb}
\usepackage{iftex}
\ifPDFTeX
  \usepackage[T1]{fontenc}
  \usepackage[utf8]{inputenc}
  \usepackage{textcomp} % provide euro and other symbols
\else % if luatex or xetex
  \usepackage{unicode-math} % this also loads fontspec
  \defaultfontfeatures{Scale=MatchLowercase}
  \defaultfontfeatures[\rmfamily]{Ligatures=TeX,Scale=1}
\fi
\usepackage{lmodern}
\ifPDFTeX\else
  % xetex/luatex font selection
\fi
% Use upquote if available, for straight quotes in verbatim environments
\IfFileExists{upquote.sty}{\usepackage{upquote}}{}
\IfFileExists{microtype.sty}{% use microtype if available
  \usepackage[]{microtype}
  \UseMicrotypeSet[protrusion]{basicmath} % disable protrusion for tt fonts
}{}
\makeatletter
\@ifundefined{KOMAClassName}{% if non-KOMA class
  \IfFileExists{parskip.sty}{%
    \usepackage{parskip}
  }{% else
    \setlength{\parindent}{0pt}
    \setlength{\parskip}{6pt plus 2pt minus 1pt}}
}{% if KOMA class
  \KOMAoptions{parskip=half}}
\makeatother
\usepackage{xcolor}
\usepackage[margin=1in]{geometry}
\usepackage{color}
\usepackage{fancyvrb}
\newcommand{\VerbBar}{|}
\newcommand{\VERB}{\Verb[commandchars=\\\{\}]}
\DefineVerbatimEnvironment{Highlighting}{Verbatim}{commandchars=\\\{\}}
% Add ',fontsize=\small' for more characters per line
\usepackage{framed}
\definecolor{shadecolor}{RGB}{248,248,248}
\newenvironment{Shaded}{\begin{snugshade}}{\end{snugshade}}
\newcommand{\AlertTok}[1]{\textcolor[rgb]{0.94,0.16,0.16}{#1}}
\newcommand{\AnnotationTok}[1]{\textcolor[rgb]{0.56,0.35,0.01}{\textbf{\textit{#1}}}}
\newcommand{\AttributeTok}[1]{\textcolor[rgb]{0.13,0.29,0.53}{#1}}
\newcommand{\BaseNTok}[1]{\textcolor[rgb]{0.00,0.00,0.81}{#1}}
\newcommand{\BuiltInTok}[1]{#1}
\newcommand{\CharTok}[1]{\textcolor[rgb]{0.31,0.60,0.02}{#1}}
\newcommand{\CommentTok}[1]{\textcolor[rgb]{0.56,0.35,0.01}{\textit{#1}}}
\newcommand{\CommentVarTok}[1]{\textcolor[rgb]{0.56,0.35,0.01}{\textbf{\textit{#1}}}}
\newcommand{\ConstantTok}[1]{\textcolor[rgb]{0.56,0.35,0.01}{#1}}
\newcommand{\ControlFlowTok}[1]{\textcolor[rgb]{0.13,0.29,0.53}{\textbf{#1}}}
\newcommand{\DataTypeTok}[1]{\textcolor[rgb]{0.13,0.29,0.53}{#1}}
\newcommand{\DecValTok}[1]{\textcolor[rgb]{0.00,0.00,0.81}{#1}}
\newcommand{\DocumentationTok}[1]{\textcolor[rgb]{0.56,0.35,0.01}{\textbf{\textit{#1}}}}
\newcommand{\ErrorTok}[1]{\textcolor[rgb]{0.64,0.00,0.00}{\textbf{#1}}}
\newcommand{\ExtensionTok}[1]{#1}
\newcommand{\FloatTok}[1]{\textcolor[rgb]{0.00,0.00,0.81}{#1}}
\newcommand{\FunctionTok}[1]{\textcolor[rgb]{0.13,0.29,0.53}{\textbf{#1}}}
\newcommand{\ImportTok}[1]{#1}
\newcommand{\InformationTok}[1]{\textcolor[rgb]{0.56,0.35,0.01}{\textbf{\textit{#1}}}}
\newcommand{\KeywordTok}[1]{\textcolor[rgb]{0.13,0.29,0.53}{\textbf{#1}}}
\newcommand{\NormalTok}[1]{#1}
\newcommand{\OperatorTok}[1]{\textcolor[rgb]{0.81,0.36,0.00}{\textbf{#1}}}
\newcommand{\OtherTok}[1]{\textcolor[rgb]{0.56,0.35,0.01}{#1}}
\newcommand{\PreprocessorTok}[1]{\textcolor[rgb]{0.56,0.35,0.01}{\textit{#1}}}
\newcommand{\RegionMarkerTok}[1]{#1}
\newcommand{\SpecialCharTok}[1]{\textcolor[rgb]{0.81,0.36,0.00}{\textbf{#1}}}
\newcommand{\SpecialStringTok}[1]{\textcolor[rgb]{0.31,0.60,0.02}{#1}}
\newcommand{\StringTok}[1]{\textcolor[rgb]{0.31,0.60,0.02}{#1}}
\newcommand{\VariableTok}[1]{\textcolor[rgb]{0.00,0.00,0.00}{#1}}
\newcommand{\VerbatimStringTok}[1]{\textcolor[rgb]{0.31,0.60,0.02}{#1}}
\newcommand{\WarningTok}[1]{\textcolor[rgb]{0.56,0.35,0.01}{\textbf{\textit{#1}}}}
\usepackage{graphicx}
\makeatletter
\newsavebox\pandoc@box
\newcommand*\pandocbounded[1]{% scales image to fit in text height/width
  \sbox\pandoc@box{#1}%
  \Gscale@div\@tempa{\textheight}{\dimexpr\ht\pandoc@box+\dp\pandoc@box\relax}%
  \Gscale@div\@tempb{\linewidth}{\wd\pandoc@box}%
  \ifdim\@tempb\p@<\@tempa\p@\let\@tempa\@tempb\fi% select the smaller of both
  \ifdim\@tempa\p@<\p@\scalebox{\@tempa}{\usebox\pandoc@box}%
  \else\usebox{\pandoc@box}%
  \fi%
}
% Set default figure placement to htbp
\def\fps@figure{htbp}
\makeatother
\setlength{\emergencystretch}{3em} % prevent overfull lines
\providecommand{\tightlist}{%
  \setlength{\itemsep}{0pt}\setlength{\parskip}{0pt}}
\setcounter{secnumdepth}{-\maxdimen} % remove section numbering
\renewcommand{\contentsname}{Índice}
\usepackage{booktabs}
\usepackage{longtable}
\usepackage{array}
\usepackage{multirow}
\usepackage{wrapfig}
\usepackage{float}
\usepackage{colortbl}
\usepackage{pdflscape}
\usepackage{tabu}
\usepackage{threeparttable}
\usepackage{threeparttablex}
\usepackage[normalem]{ulem}
\usepackage{makecell}
\usepackage{xcolor}
\usepackage{bookmark}
\IfFileExists{xurl.sty}{\usepackage{xurl}}{} % add URL line breaks if available
\urlstyle{same}
\hypersetup{
  pdftitle={ANALISIS EXPLORATORIO 1 CONCESIONARIO DE AUTOS},
  pdfauthor={Alejandra / Giovanny Porras},
  hidelinks,
  pdfcreator={LaTeX via pandoc}}

\title{ANALISIS EXPLORATORIO 1 CONCESIONARIO DE AUTOS}
\author{Alejandra / Giovanny Porras}
\date{2025-05-24}

\begin{document}
\maketitle

{
\setcounter{tocdepth}{2}
\tableofcontents
}
\subsection{2 Exploracion de datos}\label{exploracion-de-datos}

\begin{enumerate}
\def\labelenumi{\alph{enumi}.}
\tightlist
\item
  Descargar el archivo TABLA\_TALLER.xlsx
\item
  Cargar el archivo de datos en RStudio
\end{enumerate}

\textbf{Rta:} Carga de datos inicial

\begin{Shaded}
\begin{Highlighting}[]
\NormalTok{datos\_base }\OtherTok{\textless{}{-}}\NormalTok{ datos }\OtherTok{\textless{}{-}} \FunctionTok{read\_excel}\NormalTok{(}\StringTok{"D:/MaestriaAnalitica/BasesAnalitica/gitRepository/Proyecto{-}02{-}Autos/BASE/t1fe{-}tabla\_taller.xlsx"}\NormalTok{)}
\FunctionTok{kable}\NormalTok{(}\FunctionTok{head}\NormalTok{(datos\_base, }\DecValTok{10}\NormalTok{,  }\AttributeTok{caption =} \StringTok{"Datos iniciales"}\NormalTok{), }\AttributeTok{format =} \StringTok{"latex"}\NormalTok{, }\AttributeTok{booktabs =} \ConstantTok{TRUE}\NormalTok{) }\SpecialCharTok{\%\textgreater{}\%} 
  \FunctionTok{kable\_styling}\NormalTok{(}\AttributeTok{latex\_options =} \FunctionTok{c}\NormalTok{(}\StringTok{"scale\_down"}\NormalTok{, }\StringTok{"hold\_position"}\NormalTok{))}
\end{Highlighting}
\end{Shaded}

\begin{table}[!h]
\centering
\resizebox{\ifdim\width>\linewidth\linewidth\else\width\fi}{!}{
\begin{tabular}{lllllllrl}
\toprule
PERSONA & EDAD & SEXO & ESTATURA & NIVEL ESCOLAR & MARCA DE AUTO & NUMERO DE HIJOS & SALARIO & MASCOTA\\
\midrule
NA & NA & NA & NA & NA & NA & NA & NA & NA\\
NA & NA & NA & NA & NA & NA & NA & NA & NA\\
PERSONA 1 & 21 & M & 1.54 & MAESTRÍA & AUDI & 0 & 1200000 & SI\\
PERSONA 2 & 26 & F & 1.55 & PROFESIONAL & RENAULT & 5 & 1250000 & NO\\
PERSONA 3 & 30 & F & 1.6 & DOCTORADO & BMW & 2 & 900000 & NO\\
\addlinespace
PERSONA 4 & 31 & f & 1.7 & PROFESIONAL & RENAULT & 2 & 800000 & NO\\
PERSONA 5 & 35 & M & 1.71 & MAESTRÍA & AUDI & 1 & 950000 & NO\\
PERSONA 6 & 65 & M & 1.8 & MAESTRÍA & AUDI & 1 & 2000000 & SI\\
PERSONA 7 & 45 & M & 1.54 & MAESTRÍA & BMW & 1 & 2500000 & NO\\
PERSONA 8 & 42 & F & 1.52 & PROFESIONAL & RENAULT & 1 & 3500000 & SI\\
\bottomrule
\end{tabular}}
\end{table}

\begin{enumerate}
\def\labelenumi{\alph{enumi}.}
\setcounter{enumi}{2}
\tightlist
\item
  Describir brevemente la estructura del conjunto de datos: ¿Cuantos
  clientes estan registrados y que variables incluyen?
\end{enumerate}

\begin{Shaded}
\begin{Highlighting}[]
\NormalTok{no\_datos\_persona }\OtherTok{\textless{}{-}}\NormalTok{ datos\_base }\SpecialCharTok{\%\textgreater{}\%}
  \FunctionTok{filter}\NormalTok{(}\FunctionTok{grepl}\NormalTok{(}\StringTok{"PERSONA"}\NormalTok{, PERSONA)) }\SpecialCharTok{\%\textgreater{}\%}
  \FunctionTok{count}\NormalTok{()}

\NormalTok{cabeceras\_datos }\OtherTok{\textless{}{-}} \FunctionTok{names}\NormalTok{(datos\_base)}
\end{Highlighting}
\end{Shaded}

\textbf{Rta:} El conjunto de datos tiene 60 clientes registrados e
incluyen las varibles PERSONA, EDAD, SEXO, ESTATURA, NIVEL ESCOLAR,
MARCA DE AUTO, NUMERO DE HIJOS, SALARIO, MASCOTA .

\begin{enumerate}
\def\labelenumi{\alph{enumi}.}
\setcounter{enumi}{3}
\tightlist
\item
  Realizar una exploracion rapida utilizando funciones como head(),
  tail(), str(), summary(), \textbf{Rta:} Los ultimos valores son (tail)
\end{enumerate}

\begin{Shaded}
\begin{Highlighting}[]
\NormalTok{ultimos\_datos }\OtherTok{\textless{}{-}} \FunctionTok{tail}\NormalTok{(datos\_base)}
\FunctionTok{kable}\NormalTok{(}\FunctionTok{head}\NormalTok{(ultimos\_datos, }\DecValTok{10}\NormalTok{, }\AttributeTok{caption =} \StringTok{"Datos ultima posicion"}\NormalTok{), }\AttributeTok{format =} \StringTok{"latex"}\NormalTok{, }\AttributeTok{booktabs =} \ConstantTok{TRUE}\NormalTok{) }\SpecialCharTok{\%\textgreater{}\%}\FunctionTok{kable\_styling}\NormalTok{(}\AttributeTok{latex\_options =} \FunctionTok{c}\NormalTok{(}\StringTok{"scale\_down"}\NormalTok{, }\StringTok{"hold\_position"}\NormalTok{))}
\end{Highlighting}
\end{Shaded}

\begin{table}[!h]
\centering
\resizebox{\ifdim\width>\linewidth\linewidth\else\width\fi}{!}{
\begin{tabular}{lllllllrl}
\toprule
PERSONA & EDAD & SEXO & ESTATURA & NIVEL ESCOLAR & MARCA DE AUTO & NUMERO DE HIJOS & SALARIO & MASCOTA\\
\midrule
PERSONA 55 & 30 & F & 1.54 & MAESTRÍA & CHEVROLET & 2 & 2400000 & SI\\
PERSONA 56 & 39 & M & 1.58 & MAESTRÍA & AUDI & 1 & 2600000 & NO\\
PERSONA 57 & 34 & F & 1.6 & DOCTORADO & BMW & 1 & 3500000 & SI\\
PERSONA 58 & 24 & f & 1.7 & PROFESIONAL & RENAULT & 3 & 800000 & SI\\
PERSONA 59 & 20 & M & 1.71 & MAESTRÍA & AUDI & 0 & 850000 & NO\\
\addlinespace
PERSONA 60 & 10 & M & 1.8 & PROFESIONAL & AUDI & 0 & 1000000 & NO\\
\bottomrule
\end{tabular}}
\end{table}

\textbf{Funciones adicionales:}

\begin{Shaded}
\begin{Highlighting}[]
\CommentTok{\#print(str(datos\_base))}
\CommentTok{\#print(dim(datos\_base))}
\CommentTok{\#print(colnames(datos\_base))}
\FunctionTok{print}\NormalTok{(}\FunctionTok{summary}\NormalTok{(datos\_base))}
\end{Highlighting}
\end{Shaded}

\begin{verbatim}
##    PERSONA              EDAD               SEXO             ESTATURA        
##  Length:62          Length:62          Length:62          Length:62         
##  Class :character   Class :character   Class :character   Class :character  
##  Mode  :character   Mode  :character   Mode  :character   Mode  :character  
##                                                                             
##                                                                             
##                                                                             
##                                                                             
##  NIVEL ESCOLAR      MARCA DE AUTO      NUMERO DE HIJOS       SALARIO       
##  Length:62          Length:62          Length:62          Min.   : 800000  
##  Class :character   Class :character   Class :character   1st Qu.:2000000  
##  Mode  :character   Mode  :character   Mode  :character   Median :3450000  
##                                                           Mean   :3286667  
##                                                           3rd Qu.:4700000  
##                                                           Max.   :6500000  
##                                                           NA's   :2        
##    MASCOTA         
##  Length:62         
##  Class :character  
##  Mode  :character  
##                    
##                    
##                    
## 
\end{verbatim}

\begin{enumerate}
\def\labelenumi{\alph{enumi}.}
\setcounter{enumi}{4}
\tightlist
\item
  Identificar si hay datos faltantes y cuantificar cuantos son en toal y
  por variable
\end{enumerate}

\begin{Shaded}
\begin{Highlighting}[]
\CommentTok{\#is.na(datos\_base)}
\NormalTok{total\_na }\OtherTok{=}\NormalTok{ datos\_base }\SpecialCharTok{\%\textgreater{}\%}\NormalTok{ is.na }\SpecialCharTok{\%\textgreater{}\%} \FunctionTok{sum}\NormalTok{()  }
\CommentTok{\#Contar NA por columna(variable) }
\NormalTok{na\_por\_columna }\OtherTok{\textless{}{-}} \FunctionTok{colSums}\NormalTok{(}\FunctionTok{is.na}\NormalTok{(datos\_base))}
\NormalTok{tabla\_na }\OtherTok{\textless{}{-}} \FunctionTok{data.frame}\NormalTok{(}
   \AttributeTok{Variable =} \FunctionTok{names}\NormalTok{(na\_por\_columna), }
   \AttributeTok{total\_na =} \FunctionTok{as.vector}\NormalTok{(na\_por\_columna) }
\NormalTok{)}

\CommentTok{\#Contar NA total }
\CommentTok{\#rowSums(is.na(datos\_base))}

\CommentTok{\#Impresion de tabla }
\CommentTok{\#kable(tabla\_na, caption = "Variables faltantes por cuantificar")  \%\textgreater{}\%}
\CommentTok{\#  kable\_styling(full\_width = FALSE, position = "left")}
\end{Highlighting}
\end{Shaded}

\textbf{Rta}: - El numero de total de datos faltantes es \textbf{24} y
por variable son los siguientes:

\begin{longtable}[l]{lr}
\caption{\label{tab:tabla_variables_na}Variables faltantes por cuantificar}\\
\toprule
Variable & total\_na\\
\midrule
PERSONA & 2\\
EDAD & 2\\
SEXO & 3\\
ESTATURA & 2\\
NIVEL ESCOLAR & 3\\
\addlinespace
MARCA DE AUTO & 4\\
NUMERO DE HIJOS & 3\\
SALARIO & 2\\
MASCOTA & 3\\
\bottomrule
\end{longtable}

\begin{itemize}
\tightlist
\item
  Analisis: Hay problemas de datos en todas las variables sera
  importante discriminar cada caso
\end{itemize}

\begin{enumerate}
\def\labelenumi{\alph{enumi}.}
\setcounter{enumi}{5}
\tightlist
\item
  Comentar sobre los posibles problemas en los datos:
\end{enumerate}

\textbf{Filas vacías al inicio del archivo:} - Las dos primeras filas
están vacías o no contienen datos válidos.

\textbf{Inconsistencias categóricas:}

\begin{itemize}
\tightlist
\item
  Variabilidad en la codificación de la variable SEXO, con valores como
  f, mujer, hombre, nan o minúsculas inconsistentes.
\item
  Formatos no estandarizados en NIVEL ESCOLAR, como el uso de PhD en
  lugar de DOCTORADO.
\item
  Uso de minúsculas en valores de MARCA DE AUTO, como renault.
\end{itemize}

\textbf{Valores faltantes:}

\begin{itemize}
\tightlist
\item
  Algunas filas tienen múltiples variables vacías, como en el caso de la
  PERSONA 24.
\end{itemize}

\textbf{Valores extremos o anómalos (outliers):}

\begin{itemize}
\tightlist
\item
  PERSONA 31: valor de ESTATURA = 3.45 m, fuera del rango fisiológico
  normal.
\item
  PERSONA 33: valor de NUMERO DE HIJOS = 54, ampliamente fuera del
  promedio observado (3.03).
\end{itemize}

\newpage

\subsection{3 Analisis de la variable Marca ``Marca de
auto''.}\label{analisis-de-la-variable-marca-marca-de-auto.}

\begin{enumerate}
\def\labelenumi{\alph{enumi}.}
\tightlist
\item
  Evaluar la variable \textbf{``MARCA DE AUTO''} y determinar si hay
  faltantes. \textbf{Rta:}
\end{enumerate}

\end{document}
